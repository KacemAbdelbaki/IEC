\chapter*{Introduction générale}
\addcontentsline{toc}{chapter}{Introduction générale} % to include the introduction to the table of content
\markboth{Introduction générale}{} %To redefine the section page head

%Le développement technologique a transformé de nombreux secteurs, en offrant des solutions innovantes pour améliorer l'efficacité, la précision et la gestion des ressources. Parmi ces innovations, le développement mobile a pris une place prépondérante, en facilitant l'accès à des outils et des informations directement depuis nos appareils mobiles.
%\newline Cette évolution a permis aux entreprises de s'adapter aux besoins modernes, en proposant des applications qui répondent à des défis spécifiques tout en améliorant les processus internes. 

%C’est dans ce contexte d’évolution numérique que s’inscrit mon projet de stage d'immersion d'entreprise chez \textbf{Actia Engineering Services}. L'objectif principal de ce projet a été de développer une application mobile dédiée à la gestion du matériel de l’entreprise. Cette application devait non seulement automatiser certaines tâches administratives, mais aussi centraliser et rendre facilement accessibles les informations critiques liées aux équipements, contribuant ainsi à une meilleure gestion des ressources et à une organisation optimisée des processus internes.

%\begin{mylist}
%    \item \textbf{Chapitre 1 : Contexte général du projet :} 
% Dans ce premier chapitre, nous introduisons le cadre général du projet, à savoir la présentation de l'entreprise d'accueil, la présentation du projet ainsi que la méthodologie adoptée pour sa gestion.

%\item \textbf{Chapitre 2 : Analyse et spécification des besoins :}
% Ce chapitre décrit les besoins fonctionnels et non fonctionnels de l'application et présente le diagramme de cas d’utilisation. Nous clôturerons ce chapitre par présenter les outils et technologies utilisés pour le développement.

%\item \textbf{Chapitre 3 : Conception :}
%Ce chapitre décrit l’architecture fonctionnelle et la conception détaillée de la solution.

%\item \textbf{Chapitre 4 : Réalisation :}
%Enfin, le quatrième chapitre se focalise sur la réalisation et la mise en œuvre de la solution et des des captures d’écran qui illustrent les résultats obtenus.  
%\end{mylist}
%Le rapport se termine par une conclusion complète qui résume les principales conclusions et idées.
the contents of this page have been commente for my own made up reasons: to see how the page looks like lol


