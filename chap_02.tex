\chapter{Context}

\section{Introduction}
This chapter provides the necessary background to understand the project. We first describe Interactive Evolutionary Computation (IEC) and its limitations. Then, we explain how eye-tracking technology works and which metrics are useful for predicting human preference. Finally, we introduce the machine learning algorithms used in this study: Support Vector Machines (SVM) and Gradient Boosting.

\section{Interactive Evolutionary Computation (IEC)}
Evolutionary Computation is a family of algorithms inspired by biological evolution. It uses mechanisms like reproduction, mutation, and selection to find the best solution to a problem.

\subsection{The Role of the Human}
In standard optimization, a mathematical formula calculates how "good" a solution is. However, in fields like art, music, or design, this formula does not exist.
In **Interactive** Evolutionary Computation (IEC), a human user takes the role of the fitness function. The user looks at the solutions (for example, images on a screen) and selects the ones they prefer.

\subsection{The Fatigue Problem}
The main disadvantage of IEC is "User Fatigue."
\begin{itemize}
    \item \textbf{Cognitive Fatigue:} The brain gets tired of evaluating similar images over and over.
\end{itemize}
This fatigue causes the user to make mistakes or stop the process too early. To fix this, we need a system that can understand the user's preference without requiring constant manual input.

\section{Eye-Tracking Technology}
Eye-tracking is the process of measuring where a person is looking ("point of gaze") and the motion of an eye relative to the head. We use this technology to create an "Implicit" evaluation.

\subsection{Key Metrics}
Our system, the Tobii Pro Nano, captures several types of data:
\begin{enumerate}
    \item \textbf{Fixations:} Times when the eye is effectively still and processing information. A longer fixation often means more interest or deeper cognitive processing.
    \item \textbf{Saccades:} Rapid movements between fixations. The speed and path of a saccade can indicate search efficiency.
    \item \textbf{Pupil Diameter:} The size of the pupil changes not just with light, but with emotional response and mental effort (cognitive load).
\end{enumerate}

\section{Learning to Rank}
The goal of this project is to rank images from "Best" to "Worst." This is different from standard classification (which just asks "Is this a cat or a dog?"). We use specific "Learning to Rank" algorithms.

\subsection{Support Vector Machines (SVM)}
A Support Vector Machine is a supervised learning model usually used for classification. It tries to find a hyperplane (a line in 3D) that separates two classes of data with the widest possible margin.
For ranking, we use a technique called **Ranking SVM**. Instead of classifying an image as "Good" or "Bad," we classify the \textit{difference} between two images. If the model predicts the difference is positive, the first image is ranked higher.

\subsection{Gradient Boosting (LightGBM)}
Gradient Boosting is a machine learning technique that builds a prediction model in the form of an ensemble of weak prediction models, typically decision trees.
\textbf{LightGBM} is a fast and efficient version of this. It is particularly good for ranking tasks because it supports the \textbf{LambdaRank} objective. This method optimizes the order of items directly, ensuring that the best items appear at the top of the list, which is exactly what we need for our Evolutionary Algorithm.

\section*{Conclusion}
We have seen that IEC is powerful but limited by human fatigue. Eye-tracking offers a way to capture user preference implicitly. By combining this data with ranking algorithms like SVM and LightGBM, we aim to build a surrogate model that can predict user choices and automate the optimization loop.