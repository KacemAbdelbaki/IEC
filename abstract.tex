\chapter*{\huge Abstract}
\thispagestyle{empty}
\begingroup

\noindent\rule[2pt]{\textwidth}{0.5pt}\\
{\textbf{Abstract---}}
During my internship at Huawei, my mission was to develop a web application aimed at automating the process of verifying customer eligibility for ISP (Internet Service Provider) offers. The main goal of this project was to significantly reduce service delivery times by replacing manual on-site checks with a digital system leveraging existing data.

My primary tasks included analyzing client requirements, designing and developing the application using a microservices architecture with Spring Boot and Angular. I integrated key features such as centralized data management and an authentication and routing system using Eureka Server and an API Gateway.

In parallel, I set up the database infrastructure on two EulerOS instances, configuring GaussDB in a primary/standby architecture to ensure high availability and fault tolerance.

The result of this project is a fully functional web application with a user-friendly interface and features tailored to the project’s specific needs, reducing processing time and improving customer satisfaction.
%\newline
%{\textbf{Key words :}}
%Spring Boot, Angular, Microservices, Eureka, API Gateway, GaussDB, EulerOS, High Availability
%\\
\noindent\rule[2pt]{\textwidth}{0.5pt}\\[0.5cm]

{\textbf{Résumé ---}}
Durant mon stage chez Huawei, j'ai eu pour mission de développer une application web visant à automatiser le processus de vérification de l'éligibilité des clients aux offres de services FAI (Fournisseur d'Accès Internet). L'objectif principal de ce projet était de réduire considérablement le délai de fourniture du service en remplaçant les vérifications manuelles sur terrain par un système numérique exploitant les données existantes.

Mes missions ont consisté à analyser les besoins du client, concevoir et développer l'application en architecture microservices avec Spring Boot et Angular. J'ai intégré des fonctionnalités clés telles que la consultation de l'éligibilité, la gestion centralisée des données, et un système d'authentification et de routage via un serveur Eureka et une passerelle API.

En parallèle, j'ai mis en place l'infrastructure de base de données sur deux instances EulerOS, configurées avec GaussDB dans une architecture primaire/secondaire (standby) afin d'assurer la haute disponibilité et la tolérance aux pannes.

Le résultat de ce projet est une application web fonctionnelle, avec une interface conviviale et des fonctionnalités adaptées aux besoins spécifiques du projet, permettant de réduire le temps de traitement et d'améliorer la satisfaction client.

%{\textbf{Mots clés :}}
%Spring Boot, Angular, Microservices, Eureka, API Gateway, GaussDB, EulerOS, Haute disponibilité
%\\
%\noindent\rule[2pt]{\textwidth}{0.5pt}\\[1cm]
\newpage
% \begin{figure}[!ht]
% \centering
% \includegraphics[height=5.5cm]{Images/resumeee.png}
% \end{figure}
% \noindent\rule[2pt]{\textwidth}{0.5pt} 
\endgroup
% \newpage




