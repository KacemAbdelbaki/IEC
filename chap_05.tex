\chapter{Discussion and Perspectives}

\section{Introduction}
In the previous chapter, we presented the experimental results, highlighting the superior performance of the LightGBM model ($\tau \approx 0.97$) compared to the Ranking SVM ($\tau \approx 0.89$) and the Heuristic Baseline ($\tau \approx 0.36$).

In this final chapter, we interpret these findings in the broader context of Interactive Evolutionary Computation (IEC). We specifically address the research question regarding user interface design, discuss the robustness of our validation strategy, and outline the necessary steps to transition this "Offline" feasibility study into a "Online" real-time system.

\section{Interpretation of Results}
The primary objective was to determine if an implicit surrogate model could accurately replace human evaluation. The results provide a decisive validation of this hypothesis.

\subsection{Non-Linearity of User Preference}
The poor performance of the Heuristic Baseline confirms that human gaze behavior is fundamentally non-linear. A user does not simply "look longer at what they like." Complex interactions—such as a short fixation accompanied by a rapid change in pupil diameter—often signal preference more reliably than duration alone.
Tree-based models like LightGBM excel at capturing these conditional dependencies, which explains their dominance over the linear baseline and the kernel-based SVM.

\section{Implications for User Interface Design}
A key requirement of this project was to determine the most effective interaction technique for the user: presenting solutions simultaneously (Grid View) or in small groups (Pairwise View). While our experiments focused on machine learning algorithms, the results offer direct evidence to support a specific UI design.

\subsection{Algorithmic Support for Grid Views}
We tested two distinct learning approaches that mirror these UI paradigms:
\begin{itemize}
    \item \textbf{Pairwise Approach (SVM):} This models preference as a series of binary choices ($A > B$). It corresponds to a UI where users compare two images side-by-side.
    \item \textbf{Listwise Approach (LightGBM):} This models preference as a relative ranking of an entire group. It corresponds to a Grid UI where the user scans the whole population.
\end{itemize}

While the Pairwise SVM performed well ($\tau \approx 0.89$), it was outperformed by the Listwise LightGBM ($\tau \approx 0.97$). The success of the Listwise model proves that the gaze data contains sufficient signal to rank an entire generation at once.
\textbf{Conclusion:} A \textbf{Grid View UI} is scientifically viable. We do not need to restrict the user to tedious 2-by-2 comparisons to obtain accurate data. A Grid View is therefore recommended, as it allows for faster evaluation and reduces physical fatigue (fewer clicks) compared to a Pairwise interface.

\section{Robustness and Validation}
To ensure our results were not due to overfitting, we employed a rigorous two-tier validation strategy specifically designed for evolutionary data.

\subsection{Temporal Generalization}
By strictly separating the Training Set (Past Generations) from the Test Set (Future Generations), we simulated the "Time Arrow" of a real optimization session. The high accuracy on the test set proves that the model can handle \textbf{Concept Drift}—it successfully predicts future preferences based solely on past interactions.

\subsection{Subject Independence}
Through our \textbf{Leave-Subjects-Out Cross-Validation}, we observed that the model maintains high accuracy even for users it has never seen before. This implies that there are universal gaze patterns (e.g., pupil dilation upon interest) shared across different individuals, making the system scalable to new users without extensive re-calibration.

\section{Future Work}
Based on these findings, we propose the following roadmap.

\subsection{Real-Time Integration}
The immediate next step is to embed the trained LightGBM model into the IEC software loop.
\begin{enumerate}
    \item \textbf{Observation Phase:} The user views the population grid for 5--10 seconds.
    \item \textbf{Implicit Ranking:} The model predicts the fitness of all individuals instantly.
    \item \textbf{Evolution:} The algorithm generates the next generation automatically.
\end{enumerate}

% \subsection{Hybrid Optimization}
% To prevent model drift over long sessions, a \textbf{Hybrid Approach} is recommended. The system could run implicitly for several generations, but interrupt every $N^{th}$ generation to ask the user for a manual validation (explicit click). This "Human-in-the-loop" reinforcement would serve to recalibrate the model dynamically.

\section*{Conclusion}
We have established that implicit gaze analysis is a powerful and robust tool for evolutionary computation. By validating the Listwise approach, we have provided the evidence needed to design efficient Grid-based interfaces that can theoretically run optimization tasks for longer periods without exhausting the user.